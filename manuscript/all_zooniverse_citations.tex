%%\documentclass[12pt,preprint]{aastex}

%% manuscript produces a one-column, double-spaced document:

\documentclass[twocolumn]{aastex6}

\usepackage{natbib}
\usepackage{enumerate}
\bibliographystyle{apj}

%% preprint2 produces a double-column, single-spaced document:

%% \documentclass[preprint2]{aastex}
%% \documentclass[preprint2,longabstract]{aastex}

\newcommand{\vdag}{(v)^\dagger}
\newcommand{\myemail}{willett@physics.umn.edu}

%% You can insert a short comment on the title page using the command below.

\slugcomment{Not to appear in Nonlearned J., 45.}

\shorttitle{Zooniverse: Web-based Citizen Science Projects}
\shortauthors{Willett}

\begin{document}

\title{Zooniverse: Web-based Citizen Science Projects}

\author{Kyle W. Willett}
\affil{School of Physics and Astronomy, University of Minnesota,
    Minneapolis, MN 55455}

\begin{abstract}
This is a preliminary report on citizen science accounts.
\end{abstract}

\keywords{citizen science}

\section{Introduction}

The Zooniverse citizen science platform has produced dozens of peer-reviewed papers in a wide variety of fields based on crowdsourced data. Overall descriptions of the Zooniverse and citizen science are given in \citet{cle11,for12,mas13,sim14a,mar15}.

\section{Space}

\subsection{Galaxy Zoo}

\subsubsection{Galaxy Zoo 1}

The original Galaxy Zoo project is described in \citet{lin08}, with the catalog release in \citet{lin11}. The techniques for the classification debiasing are described in detail in \citet{bam09}. 

Papers using Galaxy Zoo data for population studies include those of morphology and colour on environment \citep{bam09,ski09,hoy12}, AGN \citep{sch10a}, and mergers \citep{dar10,dar10a,dar11b,ten12}. We have used Galaxy Zoo to identify unusual populations of red spiral \citep{mas10a,cor12} and blue elliptical \citep{sch09} galaxies. \citet{won12} examine the low-mass end of post-starburst galaxies. \citet{toj13} analyze star-formation histories of galaxies as a function of color and morphology. \citet{mas10b} use Galaxy Zoo data to study dust in spiral galaxies. \citet{sch14,sme15} look at star-formation histories and their effects on morphology.

Hanny's Voorwerp is a quasar light-echo discovered in the very early days of reviewing images \citep{lin09,joz09,ram10,sch10,kee12}. A further class of similar objects, called Voorwepjes, appear in \citet{kee12b,kee15}. A unique class of compact star-forming galaxies are the Green Peas, discovered by \citep{car09} and further studied by \citet{amo10,cha12} and \citet{haw12}. Orientation data of spiral galaxies were used by \citet{lan08}, \citet{slo09} and \citet{jim10}. 

External projects using Galaxy Zoo data include studies of polar ring galaxies identified in the forum \citep{fin12}, and overlapping galaxies in \citet{kee13,kee14}. Photometric redshifts as a function of morphology were studied by \citet{way11}. \citet{rod13} studied the intrinsic shapes of galaxies in SDSS.

\citet{ban10} reproduce the morphologies from Galaxy Zoo with a machine-learning algorithm. 

\subsubsection{Galaxy Zoo 2}

The Galaxy Zoo 2 (GZ2) project paper and data release is \citet{wil13}.

We used GZ2 data to study bars in \citet{hoy11,mas11c,mas12a,che13,gal15}. Dust lanes in early-type galaxies appear in \citet{kav12a,sha12b}; this is also related to tidal dwarf galaxies \citep{kav12} and spheroidal post-mergers \citep{car12,kav13a}. \citet{ski12} studied the environmental dependence of bars and bulges in disc galaxies. \citet{cas13} quantify morphological indicators of galaxy interaction, centering on loosely-wound spiral arms. \citet{sim13} use GZ2 galaxies to identify a bulgeless population with actively growing black holes. \citet{wil15} studies the effect of disk morphology on the SF-stellar~mass relation.

\citet{dav12a,dav14} trained an algorithm to detect spiral-arm galaxy structure calibrated using GZ2 data. \citet{die15} developed a convolutional neural network to replicate the results of the GZ2 catalog in conjuction with a Kaggle competition.

\subsubsection{Galaxy Zoo: Hubble}

The Galaxy Zoo: Hubble (GZH) project paper and data release is Willett et al. (2016, in prep).

\citet{mel14} used early GZH data to measure the declining bar fraction as a function of redshift. \citet{che15} investigate the relationship between bars and AGN over the various surveys in GZH.

\subsubsection{Galaxy Zoo: CANDELS}

The Galaxy Zoo: CANDELS (GZC) project paper and data release is Simmons et al. (2016, submitted).

\citet{sim14} use early GZC data to measure the bar fraction at $z\sim2$.

\subsection{Galaxy Zoo: Supernova}

The Galaxy Zoo: Supernova project was described in \citet{smi11}, and an improved dynamic Bayesian classification system used for data analysis in \citet{sim13a}. GZ: Supernova results also appear in papers by \citet{mag11,mag12a} and \citet{lev13}. 

\subsection{Galaxy Zoo: Mergers}

Galaxy Zoo: Mergers was a separate project from Galaxy Zoo, using simulations to characterize the dynamical effects of major mergers \citep{hol16}.

\subsection{Planet Hunters}

Planet Hunters has identified many exoplanet candidates using public archive data from the {\it Kepler} spacecraft. Unlike other Zooniverse papers, most publications have been explicitly numbered in a series:

\begin{enumerate}[(I)]

\item \citet{fis12} - first two candidates (PH1, PH2) discovered by the project
\item \citet{sch12} - inventory of Kepler short-period planets
\item \citet{sch13} - first circumbinary planet in a quadruple star system
\item \citet{lin13} - new candidates from Kepler Quarter 2
\item \citet{wan13} - a confirmed Jupiter-size planet in the habitable zone
\item \citet{sch14c} - independent characterization of KOI-351 and other long-period candidates
\item \citet{sch14b} - discovery of new planet (PH3 c) and mass measurements of PH3 b and PH3 d
\item \citet{wan15} - measurements of 41 long-period candidates
\item \citet{boy16} - ``Tabby's Star'', a deeply unusual object with up to 20\% aperiodic dips in flux
\item \citet{sch16} - search for nearby neighbors of 75 K2 targets

\end{enumerate}

\noindent In non-numbered papers, \citet{kat14} credit Planet Hunters in helping to identify an unusually active dwarf nova. \citet{gie13} describes the discovery of a new cataclysmic variable.

\subsection{Solar Stormwatch}

Solar Stormwatch used citizen science data to track 2D structure of coronal mass ejections \citep{sav12}. They also inferred the distribution of interplanetary dust between 0.96 and 1.04~AU \citep{dav12}. Their coronal mass ejection catalog \citep{bar14} was compared against results from experts and automated algorithms in \citet{tuc14,bar15}.

\citet{wil16} combines results from Solar Stormwatch and Old Weather in a novel way of measuring historical aurora activity.

\subsection{Milky Way Project}

The Milky Way project design and data release appears in \citet{sim12a}. \citet{ken12} use this data to make a statistical study of massive star formation associated with infrared bubbles, and their proximity to cold clumps from ATLASGAL \citep{ken16}. \citet{ker15} describe the compact infrared sources dubbed ``yellowballs''.

\citet{bea14} leverage the training set of the MWP to create Brut, an automated algorithm for identifying bubbles in infrared images of the midplane.

\subsection{Space Warps}

Results and methodology for Space Warps are given by \citet{mar16} and \citet{mor16}. The project discovered the ``Red Radio Ring'', a gravitationally-lensed hyperluminous galaxy at $z=2.553$ \citep{gea15}. Users also attempted to model the mass distributions of the lensing systems \citep{kun15}.

\subsection{Radio Galaxy Zoo}

The project description for Radio Galaxy Zoo is \citet{ban15}. 

\subsection{Moon Zoo}

Preliminary results on crater counts at the Apollo~17 landing site are in \citet{bug16}.

\subsection{Snapshot Supernova}

The Snapshot Supernova project was developed for Stargazing Live in 2015. \citet{cam15,car15} made spectroscopic classifications of the newly-discovered optical transients.

\subsection{Andromeda Project}

The first catalog of stellar clusters from Andromeda is \citet{joh15}.

\subsection{Ice Hunters}

\citet{par13} discovered 2011~HM$_{102}$, a high-inclination L5 Neptune Trojan, which was recovered in a parallel effort by Ice Hunters.

\section{Nature}

\subsection{Whale FM}
\citet{say13} studied repeated call types in short-finned pilot whales using Whale FM data. \citet{sha14} analyzed whale calls using computer algorithms in attempt to accurately identify different species.

\subsection{Snapshot Serengeti}

The project description of Snapshot Serengeti is \citet{swa15}, with validation methods in \citet{swa16}. Data is used to estimate the spatial density of lions in \citet{cus15}.

\section{Climate}

\subsection{Cyclone Center}

\citet{hen14} show that the modern global tropical cyclone record is littered with uncertainty; crowdsourcing could help clean up the mess.

\section{Medicine}

\subsection{Cell Slider}

\citet{can15} describe Cell Slider and comparison of tracking estrogren receptors to experts.

\section{Humanities}

\subsection{Ancient Lives}

Ancient Lives is a humanities project transcribing Greek papyri. This included the publication of a new volume for the Oxyrhynchus Papyri \citep{bru13}. The algorithms and pipeline are described in \citet{wil14a,wil14}.

\subsection{Operation War Diary}

\citet{gra16} analyzes data for WWI British infantry and cavalry divisions, showing the difference in time spent in combat between the various divisions.

\section{Citizen-science: sociology, education research, and machine learning}

\subsection{Sociology and meta-studies}

Motivations of Zooniverse users were studied in \citet{rad10,rad13,jac15}. \citet{man11b} did an interpretive study of the meanings that citizen scientists make when using Galaxy Zoo. \citet{cox15} defines various metrics of success for citizen science projects. \citet{eve13,eve14,gre14} discuss gamification and user engagement in Old Weather. \citet{jac16} examines trajectories of newcomers to the Zooniverse. \citet{jen13} research creativity in web-based citizen science. \citet{luc14,mug14,mug15} examine online communities in Seafloor Explorer and Planet Hunters.

\citet{klo13,pra13,mas16} examines how and what volunteers learn through participation.

\citet{chr12,mad14} describe contributions of citizen science to astronomy research.

\subsection{Education research}

\citet{sla11} describe using Galaxy Zoo as a teaching tool for stimulating scientific inquiry. \citet{bor13} summarizes Zooniverse tools for formal and informal audience engagement.


\acknowledgments

{\it Facilities:} 
\facility{SDSS},
\facility{HST},
\facility{Spitzer},
\facility{Chandra},
\facility{Kepler},
\facility{STEREO},
\facility{LRO},
\facility{CFHT},
\facility{Magellan},
\facility{VLA},
\facility{ATCA},
\facility{WISE},

\bibliography{../zooniverse}
\end{document}

%%
%% End of file `sample.tex'.
